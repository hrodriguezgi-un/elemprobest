% Options for packages loaded elsewhere
\PassOptionsToPackage{unicode}{hyperref}
\PassOptionsToPackage{hyphens}{url}
%
\documentclass[
  ignorenonframetext,
]{beamer}
\usepackage{pgfpages}
\setbeamertemplate{caption}[numbered]
\setbeamertemplate{caption label separator}{: }
\setbeamercolor{caption name}{fg=normal text.fg}
\beamertemplatenavigationsymbolsempty
% Prevent slide breaks in the middle of a paragraph
\widowpenalties 1 10000
\raggedbottom
\setbeamertemplate{part page}{
  \centering
  \begin{beamercolorbox}[sep=16pt,center]{part title}
    \usebeamerfont{part title}\insertpart\par
  \end{beamercolorbox}
}
\setbeamertemplate{section page}{
  \centering
  \begin{beamercolorbox}[sep=12pt,center]{part title}
    \usebeamerfont{section title}\insertsection\par
  \end{beamercolorbox}
}
\setbeamertemplate{subsection page}{
  \centering
  \begin{beamercolorbox}[sep=8pt,center]{part title}
    \usebeamerfont{subsection title}\insertsubsection\par
  \end{beamercolorbox}
}
\AtBeginPart{
  \frame{\partpage}
}
\AtBeginSection{
  \ifbibliography
  \else
    \frame{\sectionpage}
  \fi
}
\AtBeginSubsection{
  \frame{\subsectionpage}
}
\usepackage{amsmath,amssymb}
\usepackage{lmodern}
\usepackage{ifxetex,ifluatex}
\ifnum 0\ifxetex 1\fi\ifluatex 1\fi=0 % if pdftex
  \usepackage[T1]{fontenc}
  \usepackage[utf8]{inputenc}
  \usepackage{textcomp} % provide euro and other symbols
\else % if luatex or xetex
  \usepackage{unicode-math}
  \defaultfontfeatures{Scale=MatchLowercase}
  \defaultfontfeatures[\rmfamily]{Ligatures=TeX,Scale=1}
\fi
% Use upquote if available, for straight quotes in verbatim environments
\IfFileExists{upquote.sty}{\usepackage{upquote}}{}
\IfFileExists{microtype.sty}{% use microtype if available
  \usepackage[]{microtype}
  \UseMicrotypeSet[protrusion]{basicmath} % disable protrusion for tt fonts
}{}
\makeatletter
\@ifundefined{KOMAClassName}{% if non-KOMA class
  \IfFileExists{parskip.sty}{%
    \usepackage{parskip}
  }{% else
    \setlength{\parindent}{0pt}
    \setlength{\parskip}{6pt plus 2pt minus 1pt}}
}{% if KOMA class
  \KOMAoptions{parskip=half}}
\makeatother
\usepackage{xcolor}
\IfFileExists{xurl.sty}{\usepackage{xurl}}{} % add URL line breaks if available
\IfFileExists{bookmark.sty}{\usepackage{bookmark}}{\usepackage{hyperref}}
\hypersetup{
  pdftitle={La Latitud define la temperatura de un lugar?},
  pdfauthor={Harvey Rodriguez Gil},
  hidelinks,
  pdfcreator={LaTeX via pandoc}}
\urlstyle{same} % disable monospaced font for URLs
\newif\ifbibliography
\setlength{\emergencystretch}{3em} % prevent overfull lines
\providecommand{\tightlist}{%
  \setlength{\itemsep}{0pt}\setlength{\parskip}{0pt}}
\setcounter{secnumdepth}{-\maxdimen} % remove section numbering
\ifluatex
  \usepackage{selnolig}  % disable illegal ligatures
\fi

\title{La Latitud define la temperatura de un lugar?}
\author{Harvey Rodriguez Gil}
\date{Mayo 7, 2021}

\begin{document}
\frame{\titlepage}

\hypertarget{descripciuxf3n-de-la-base-de-datos}{%
\section{Descripción de la base de
datos}\label{descripciuxf3n-de-la-base-de-datos}}

\begin{frame}{fuente}
\protect\hypertarget{fuente}{}
El archivo fuente utilizado se encuentra en la carpeta
\href{sources/daily_weather_2020.csv}{sources}, y fue originalmente
obtenido desde \textbf{\emph{Kaggle}}:
\href{https://www.kaggle.com/vishalvjoseph/weather-dataset-for-covid19-predictions}{\emph{Historical
Daily Weather Data 2020}}
\end{frame}

\begin{frame}{Descripción de la información}
\protect\hypertarget{descripciuxf3n-de-la-informaciuxf3n}{}
Este conjunto de datos contiene la información diaria del clima para 163
países (algunos con información de estados/provincias) desde el 1º de
Enero de 2020 hasta el 20 de Abril de 2020. La lista de los países fue
basado en el conjunto de datos de John Hopkins COVID-19.
\end{frame}

\hypertarget{descripciuxf3n-de-las-variables}{%
\section{Descripción de las
variables}\label{descripciuxf3n-de-las-variables}}

\begin{frame}{A continuación se describiran las variables utilizadas en
el ejercicio:}
\protect\hypertarget{a-continuaciuxf3n-se-describiran-las-variables-utilizadas-en-el-ejercicio}{}
\begin{itemize}
\item
  \textbf{Country/Region:} país a los que le corresponden los datos
\item
  \textbf{Province/State:} estado del pais al que le corresponde los
  datos.
\item
  \textbf{Time:} fecha en la que fueron tomados los datos
\item
  \textbf{Summary:} texto resumen (humano) del estado del clima.
\item
  \textbf{Icon:} texto resumen (máquina) del estado del clima. Utilizado
  para presentar un ícono que represente el estado del clima.
\item
  \textbf{Humidity:} humedad relativa
\item
  \textbf{Pressure:} Presión del aíre
\item
  \textbf{TemperatureMin:} Temperatura mínima registrada para un día
\item
  \textbf{TemperatureMax:} Temperatura máxima registrada para un día
\end{itemize}
\end{frame}

\end{document}
